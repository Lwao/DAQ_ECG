\section{Introduction}

An electrocardiograph (ECG) is an instrument responsible to record the electrical activity of the heart, considering that electrical signals are generated before the heart mechanical functions and generated by nerve impulse stimulus. In this context the ECG provides information about those signals that allows to detect a variety of cardiac disorders \cite{khandpur1987handbook, murugappan2014development}.

The ECG machine has became an essential instrument throughout the years once it is a noninvasive, simple to record and minimal cost device \cite{barold2003willem}. Regarding its importance, the objective of this work is to design an experimental setup for an ECG machine, highlighting all project steps so anyone can learn experiment with a simple and accessible device for the analysis of cardiograph signals.

The first step of the ECG design is to get to know how the cardiac signal is generated, its specifications and the noises associated with the data acquisition.

Once define those guidelines, the next step is to develop a reliable source of cardiac signals, both in the software and hardware domain, also known as a simulator. A simulator from the software point of view can be used to learn how to process the signal with multiples digital signal processing algorithms in a higher level of abstraction. From the hardware point of view there is a need to generate a cardiac signal in the analog domain so it can be acquired by the proposed data acquisition system (DAQ). For the latter a consolidated ECG device could be used, but once the project relies in the simplicity of it resources, a simple circuit serves its purposes.

The third step concerns the ECG design itself. The sub steps are:

\begin{itemize}
    \item Design the low noise preamplifier(instrumentation amplifier), since the acquired signal by the electrodes has low voltage;
    \item Design a low-pass analog filter serving as an antialising filter to avoid out of band noise contamination after the signal is digitized;
    \item Design a high-pass filter to eliminate the DC component and any baseline wander;
    \item Design a Notch filter (reject band filter) centered in 60 Hz to remove any interference from the power grid;
    \item Data acquisition by an analog-to-digital converter (ADC);
    \item Program the micro controller interface between the analog front end and the digital part responsible for analyzing, processing, storing and displaying the final graphical output.
\end{itemize}

Where the first five sub steps integrate the analog front end that is responsible for conditioning and digitizing the signal.

The final step is to validate the system with experiments and discuss the results.

\pagebreak